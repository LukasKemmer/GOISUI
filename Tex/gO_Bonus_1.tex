\documentclass[ngerman, a4paper,12pt]{article}

%\usepackage[applemac]{inputenc} % Bei Benutzung von Apple-Betriebssystemen bitte durch ``\usepackage[applemac]{inputenc}'' ersetzen.
\usepackage[utf8]{inputenc}
\usepackage[T1]{fontenc}

\usepackage[ngerman]{babel} 
\usepackage{fixltx2e}
\usepackage{tabularx}
\usepackage{booktabs}
\usepackage{placeins}
\usepackage{eurosym}
\usepackage{amssymb,amsmath}

\usepackage{graphicx} 
\usepackage{color}
\definecolor{kit}{cmyk}{1,0,0.6,0}

\usepackage{hyperref}
\hypersetup{pdftoolbar=true,
            pdfmenubar=true,
            pdfpagemode=UseOutlines,
            bookmarksnumbered=true,
            linktocpage=true,
            colorlinks=false,
            %backref, % Entkommentieren, um zu sehen, ob alle Literaturstellen im Text zitiert werden.
            colorlinks=false
            }

\newtheorem{definition}{Definition}[section]
\newtheorem{satz}[definition]{Satz}
\newtheorem{lemma}[definition]{Lemma}
\newtheorem{korollar}[definition]{Korollar}
\newtheorem{proposition}[definition]{Proposition}
\newtheorem{bemerkung}[definition]{Bemerkung}
\newtheorem{beispiel}[definition]{Beispiel}
\newtheorem{problem}[definition]{Problem}
\newtheorem{Voraussetzung}[definition]{Voraussetzung}
\newtheorem{algorithmus}[definition]{Algorithmus}
\newtheorem{vermutung}[definition]{Vermutung}

\setlength{\parindent}{0pt}
\parskip1.5ex

\newcommand{\R}{\mathbb R} % Beispiel für die Definition eines eigenen Befehls

\begin{document}

\begin{flushleft}
\vspace*{-100pt}
\textbf{Institut f\"ur Operations Research \\
Prof. Dr. Oliver Stein \\}
Sommersemester 2018
\vspace*{15pt}
\end{flushleft}

\begin{flushright}
\vspace*{-80pt}
\includegraphics[scale=0.5]{kit_logo}
\vspace*{15pt}
\end{flushright}

\begin{center}
\textbf{Erste Bonusübung zur Vorlesung \\
\emph{Globale Optimierung I}}        
\end{center}

\begin{table}[h]
	\centering
	\begin{tabularx}{\textwidth}{X X X}
		Name & Matrikelnummer & Studiengang \\
		\toprule
		Leon Qadirie					& & Wi.Ing. (M.Sc)\\
		Lukas Kemmer 				& 1725171			& Wi.Ing. (M.Sc)\\
		\bottomrule
	\end{tabularx}

\end{table}

\textbf{Aufgabe S1.1} \\
(a) Für die eindimensionale Einschränkung $\phi_d(t)$ mit $d \neq 0$ gilt
\begin{equation}
	\begin{split}
		\lim\limits_{t \rightarrow +\infty} \phi_d(t) &= \lim\limits_{t \rightarrow +\infty} 2t^4d_1^2 - 3t^3d_1d_2^2+t^2d_2 = + \infty \\
		\lim\limits_{t \rightarrow -\infty} \phi_d(t) &= \lim\limits_{t \rightarrow -\infty} 2t^4d_1^2 - 3t^3d_1d_2^2+t^2d_2 = + \infty
	\end{split}
\end{equation}
da bei der Betrachtung der Grenzwerte jeweils der Term mit dem höchsten Exponenten maßgeblich ist. Hier bedeutet dies, dass für $d_1 \neq 0$ der Grenzwert der eindimensionalen Einschränkung durch $2t^4d_1^2$ bestimt wird, welcher für $t \rightarrow \pm \infty$ gegen $+ \infty$ geht. Für $d_1=0$ folgt aus $d \neq 0$ das $d_2 \neq 0$ und damit $\phi_d(t)=t^2d_2$ was offensichtlich auch für $t \rightarrow \pm \infty$ gegen $+ \infty$ geht. Sei nun $(x^k) \subseteq \R$ eine Folge mit $lim_k ||x^k||_1 \rightarrow \infty$ dann folgt aus (1) auch
\begin{equation}
	\lim\limits_{k} \phi_d(x^k) = + \infty.
\end{equation}
Gleichung (2) gilt, da aus der Verwendung der 1-Norm und der Eindimensionalität der Folgeglieder $x^k$ folgt, dass für $||x^k||_1 \rightarrow \infty$ die Folgeglieder $x^k \rightarrow \pm \infty$ gehen müssen. Nach Definition 1.2.40 ist die eindimensionale Einschränkung $\phi_d(t)$ damit für jede Richtung $d \in \mathbb{R}^2 \backslash \{0\}$ koerziv. \par
(b) Sei $(x^k) \subseteq \mathbb{R}^2$ eine Folge mit $x^k = (k^2, k)$. Dann gilt $\lim_k ||(x^k)||_1 = |k^2|+|k| \rightarrow \infty$. Für f gilt
\begin{equation}
	\lim\limits_{k} f(x^k) = \lim\limits_{k} (2k^2-k^2)(k^2-k^2) = 0,
\end{equation}
und damit dass eine Folge $(x^k) \in \mathbb{R}^2$ existiert mit $lim_k ||x^k||_1 \rightarrow \infty$ die die Bedingung $\lim_k f(x^k) 0 + \infty$ aus Definition 1.2.40 verletzt. Damit ist $f$ nicht koerziv. \par
(c) Sei $f: M \rightarrow Y$ eine koerzive Funktion auf $M$ mit $Y \subseteq \mathbb{R}$. Dann gilt für eine Folge $(x^k) \subseteq M$ mit $\lim_{k} \Vert x^k \| = \infty$ nach Definition 1.2.40 auch
\begin{equation*}
\lim\limits_{k} f(x^k) = + \infty .
\end{equation*}
Sei nun $\psi : Y \rightarrow \mathbb{R}$  mit
\begin{equation*}
\psi (y) = 1- e^{-y}
\end{equation*}
dann gilt
\begin{equation*}
\psi ' (y) = e^{-y} > 0
\end{equation*}
und somit, dass $\psi$ streng monoton wachsend ist. Weiterhin gilt 
\begin{equation*}
\lim_{y \rightarrow \infty} \psi(y) = \lim_{y \rightarrow \infty} 1-e^{-y} = 1.
\end{equation*}
Für die Verkettung $\psi \circ f$ folgt nun
\begin{equation*}
\lim\limits_{k} \psi(f(x^k)) = 1
\end{equation*}
und damit insbesondere, dass die Definition 1.2.40 nicht erfüllt und somit $\psi \circ f$ nicht koerziv ist. Damit ist ein Gegenbeispiel gefunden und die Aussage ist falsch. \par
\textbf{Aufgabe S1.2} \\
(a) Sei $F: \mathbb{R}^m \rightarrow \mathbb{R}$ mit $F(y) = \sum_{i=1}^{m}y_i$, $\beta = (a, b)^T \in \mathbb{R}^{n+1}$ sowie $f: \mathbb{R}^{n+1} \rightarrow \mathbb{R}^m$ mit 
\begin{equation*}
	f(\beta) = 
					\begin{pmatrix}
					\left| (x^1)^Ta + b - y_1 \right| \\
					... \\
					\left| (x^m)^Ta + b - y_m \right| \\
					 \end{pmatrix} =
											 \begin{pmatrix}
											 \max\{(x^1)^Ta + b - y_1 , y_1 - (x^1)^Ta - b\} \\
											 ... \\
											 \max\{(x^m)^Ta + b - y_m , y_m - (x^m)^Ta - b\} \\
											 \end{pmatrix}.
\end{equation*}
Das Problem $P1$ kann dann geschrieben werden als
\begin{equation*}
	P_1: \min_{\beta} F(f(\beta)).
\end{equation*}
Mithilfe der verallgemeinmerten Epigraph-Umformung ergibt sich analog zu Übung 3.5 das äquivalente Optimierungsproblem
\begin{equation*}
	P_{1, epi}:\min_{\alpha \in \mathbb{R}^m, \beta \in \mathbb{R}^{n+1}} F(\alpha) \ s.t. \ f(\beta) \leq \alpha.
\end{equation*}
Da jede Nebenbedingung $\max\{z_1, z_2\} \leq \gamma$ in zwei Nebenbedingungen $z_1 \leq \gamma, \ z_2 \leq \gamma$ umgeformt werden kann, kann $P_{epi}$ in das äquivalente lineare Optimierungsproblem 
\begin{equation*}
	\begin{split}
	P_{1, lin}: \min_{\alpha \in \mathbb{R}^m, \beta \in \mathbb{R}^{n+1}} & \sum_{i=1}^{m} \alpha_i \\
	s.t. \ &(x^1)^Ta + b -y_1 \leq \alpha_1 \\
	& ... \\
	& (x^m)^Ta + b -y_m \leq \alpha_m \\
	& y_1 - (x^1)^Ta - b \leq \alpha_1 \\
	& ... \\
	& y_m - (x^m)^Ta - b \leq \alpha_m 
	\end{split}
\end{equation*}
mit $2m$ Restriktionen umgeformt werden. \par
(b) Sei $\beta = (a, b)^T \in \mathbb{R}^{n+1}$ und $\psi: \mathbb{R}_+ \rightarrow \mathbb{R}$ mit $\psi(y)=0.5y^2$, dann ist $\psi$ wegen $\psi'(y)=y > 0$ streng monoton wachsend auf dem Intervall $\left[0, \infty \right)$. Mithilfe der Funktion $\psi$ kann $P_2$ nun durch eine monotone Transformation (siehe Buch, Übung 1.3.5) in das äquivalente Problem $P_{2, MR}$ transformiert werden mit
\begin{equation*}
	P_{2, MR}:\min_{\beta \in \mathbb{R}^{n+1}} \frac{1}{2} \|X \beta - y \|_2^2.
\end{equation*}
Für die Zielfunktion $g: \mathbb{R}^{n+1} \rightarrow \mathbb{R}$ mit $g(\beta)= \frac{1}{2} \|X \beta - y \|_2^2$ und ein $c>0$ gilt weiterhin
\begin{equation*}
\begin{split}
	\nabla (g(\beta) - \frac{c}{2} \| \beta \|_2^2) &= X^T(X\beta - y) - c\beta, \\
	H(g(\beta) - \frac{c}{2} \| \beta \|_2^2) &= X^TX - c \succeq 0,
\end{split}
\end{equation*}
da die Matrix $X$ vollen Rang hat, sofern $\exists \ i,j \in {1, ...m ,}: i \neq j \wedge x^i \neq x^j$. Damit ist $g(\beta)$ nach Definition 2.1.1 (d) gleichmäßig konvex (beachte, dass $\mathbb{R}^{n+1}$ eine konvexe Menge ist). Nach Lemma 2.3.2 folgt damit, und dem Fakt dass der $\mathbb{R}^{n+1}$ trivialerweise abgeschlossen ist, dass $P_{2, mr}$ mit Korollar 1.2.43 und Satz 1.2.5 lösbar ist. Aufgrund der Konvexität von $g$ und dem $\mathbb{R}^{n+1}$ ist $P_{2,MR}$ nach Definition 2.1.5 ein konvexes Optimierungsproblem. Aus Korollar 2.4.6 folgt, dass die Globalen Minimalpunkte von $g$ genau den kritischen Punkte entsprechen. Es folgt
\begin{equation}
	\begin{split}
		\nabla g(\beta) &= X^T(X\beta -y),\\
		0 &\overset{!}{=} X^T(X\beta -y) \\
		\iff \beta	&=	(X^TX)^{-1}X^Ty.
	\end{split}
\end{equation}
Dabei sei angemerkt, dass $X^TX$ invertierbar ist, da $X$ vollen rang hat. \par
(c) Siehe Python. \par
(d) blablabla
\end{document}